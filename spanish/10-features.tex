%\newpage
\chapter{Features}
\label{sec:features}
%\begin{enumerate}
\begin{enumerate}
\item Opera con los controladores ATmega8, ATmega168 o ATmega328. También se pueden usar los controladores ATmega1280 o ATmega2560.
\item Muestra los resultados en una pantalla LCD de 2x16 o 4x16 caracteres.
\item Una sola tecla de operación con apagado automático.
\item Se pueden usar baterías para su funcionamiento dado que el consumo es de sólo 20nA.
\item Versión de bajo coste viable sin cristal y autoapagado. Con la versión 1.05k se utiliza el modo suspendido del Atmega168 o ATmega328 para reducir el consumo si no hay medidas que realizar.
\item Detección automática de transistores bipolares NPN y PNP, MOSFETs de canales N y P, JFETs, diodos, diodos dobles, tristores y triacs.
\item Detección automática de la disposición de los pines del componente analizado.
\item Medida del factor de amplificación y del umbral base-emisor de los transistores bipolares. % Me preocupa esta línea - Ivan 21 dic 14 %
\item Se pueden identificar los transistores Darlington por el umbral de voltaje y el alto factor de amplificación.
\item Detección del diodo de protección en los transistores bipolares y MOSFETs.
\item En los MOSFETs, medida del umbral de voltaje de la Puerta y del valor de capacidad de la puerta.
\item Se miden y muestran hasta dos resistencias a la vez con sus símbolos
\setlength{\unitlength}{0.1mm}
\linethickness{0.4mm}
\begin{picture}(60,30)
\put(0,15){\line(1,0){10}}
\put(10,5){\line(0,1){20}}
\put(10,5){\line(1,0){40}}
\put(10,25){\line(1,0){40}}
\put(50,5){\line(0,1){20}}
\put(50,15){\line(1,0){10}}
\end{picture}
% and values with up to four decimal digits in the right dimension.
y valores con hasta cuatro dígitos decimales hacia la derecha.
Todos los símbolos están rodeados por los números de prueba del Tester (1-3)
De esta manera, puede medirse también un potenciómetro. Si el potenciómetro está ajustado a uno de sus extremos, 
el Tester no podrá distinguir entre el pin intermedio y el pin final.
\item La resolución para la medida de resistencias es ahora de hasta \(0.01\Omega\), y se detectan valores de hasta \(50M\Omega\)
\item Se puede detectar y medir un sólo acumulador cada vez. Se muestra con el símbolo
\setlength{\unitlength}{0.1mm}
\begin{picture}(60,30)
\linethickness{0.4mm}
\put(0,15){\line(1,0){20}}
\put(40,15){\line(1,0){20}}
\put(22,0){\line(0,1){30}}
\put(26,0){\line(0,1){30}}
\put(34,0){\line(0,1){30}}
\put(38,0){\line(0,1){30}}
\end{picture}
y valores con hasta cuatro dígitos decimales hacia la derecha.
El valor puede ser de entre \(25 pF\) (reloj de 8MHz, \(50 pF\) con reloj @1MHz) hasta \(100 mF\). La resolución puede ser de hasta \(1 pF\) (reloj @8MHz clock).
\item Para condensadores con un valor de capacidad por encima de \(0.18\mu F\) se mide
\item For capacitors with a capacity value above \(0.18 \mu F\) the Equivalent Serial Resistance (ESR) is measured 
with a resolution of \(0.01 \Omega\) and is shown with two significant decimal digits.
This feature is only avaiable for ATmega with at least 16K flash memory (ATmega168 or ATmega328).
\item For capacitors with a capacity value above \(5000 pF\) the voltage loss after a load pulse can be determined.
The voltage loss give a hint for the quality factor of the capacitor.
\item Up to two diodes are shown with symbol
\setlength{\unitlength}{0.1mm}
\begin{picture}(60,30)
\linethickness{0.4mm}
\put(0,15){\line(1,0){60}}
\put(22,2){\line(0,1){26}}
\put(26,6){\line(0,1){18}}
\put(30,10){\line(0,1){10}}
\put(38,2){\line(0,1){26}}
\end{picture}
or symbol
\setlength{\unitlength}{0.1mm}
\begin{picture}(60,30)
\linethickness{0.4mm}
\put(0,15){\line(1,0){60}}
\put(38,2){\line(0,1){26}}
\put(34,6){\line(0,1){18}}
\put(30,10){\line(0,1){10}}
\put(22,2){\line(0,1){26}}
\end{picture}
in correct order. Additionally the flux voltages are shown.
\item LED is detected as diode, the flux voltage is much higher than normal. 
Two-in-one LEDs are also detected as two diodes.
\item Zener-Diodes can be detected, if reverse break down Voltage is below 4.5V.
These are shown as two diodes, you can identify this part only by the voltages.
The outer probe numbers, which surround the diode symbols, are identical in this case.
You can identify the real Anode of the diode only by the one with break down (threshold) Voltage nearby 700mV!
\item If more than 3 diode type parts are detected, the number of founded diodes is shown additionally to the fail message.
 This can only happen, if Diodes are attached to all three probes and at least one is a Z-Diode.
In this case you should only connect two probes and start measurement again, one after the other.
\item Measurement of the capacity value of a single diode in reverse direction.
Bipolar Transistors can also be analysed, if you connect the Base and only one of Collector or Emitter.
\item Only one measurement is needed to find out the connections of a bridge rectifier.
\item Capacitors with value below 25pF are usually not detectet, but can be measured together with
a parallel diode or a parallel capacitor with at least 25pF.
In this case you must subtract the capacity value of the parallel connected part.
\item For resistors below \(2100 \Omega\) also the measurement of inductance will be done, if
your ATmega has at least 16K flash memory.
The range will be from about \(0.01 mH\) to more than \(20 H\), but the accuracy is not good.
The measurement result is only shown with a single component connected.
\item Testing time is about two seconds, only capacity or inductance measurement can cause longer period.
\item Software can be configured to enable series of measurements before power will be shut down.
\item Build in selftest function with optional 50Hz Frequency generator to check the accuracy of clock frequency and wait calls (ATmega168 and ATmega328 only).
\item Selectable facility to calibrate the internal port resistance of port output and
the zero offset of capacity measurement with the selftest (ATmega168 and ATmega328 only).
A external capacitor with a value between \(100 nF\) 
and \(20 \mu F\) connected to pin~1 and pin~3 is necessary to compensate the offset voltage of the analog comparator.
This can reduce measurement errors of capacitors of up to \(40 \mu F\).
With the same capacitor a correction voltage to the internal reference voltage is found to adjust the
gain for ADC measuring with the internal reference.
\item Display the Collector cutoff current \(I_{CE0}\) with currentless base (\(10\mu A\) units) and
Collector residual current \(I_{CES}\) with base hold to emitter level (ATmega328 only).
This values are only shown, if they are not zero (especially for Germanium transistors).
\item For the ATmega328 a dialog function can be selected, which enable additional functions.
Of course you can return from dialog to the normal Transistor Tester function.
\item With dialog function you can use a frequency measurement at port PD4 of the ATmega.
The resolution is 1~Hz for input frequencies above \(25~kHz\).
For lower frequencies the resolution can be up to \(0.001~mHz\) by measuring the mean period.
\item With the dialog function and without the activated serial output a external voltage of up to 50V can be measured with
the 10:1 voltage divider at the PC3 port. If the PLCC-Version of the ATmega328 is used, one of the additional
pins can be used for the voltage measurement together with the serial output.
If the zener diode measurement extension (DC-DC converter) is assembled, the measurement of
zener diodes is also possible with this function by pressing the key.
\item With the dialog function a frequency output can be selected at the TP2 pin (PB2 Port of the ATmega).
Currently a preselection of frequencies from 1~Hz up to 2~MHz can be selected.
\item With the dialog function a fixed frequency output with selectable pulse width can be activated at the TP2 pin
(PB2 port of the ATmega). The width can be enhanced with 1\% by a short key press or with 10\% by a longer key press.
\item With the dialog function can be started a separated capacity measurement with ESR measurement.
Capacities from about \(2 \mu F\) up to \(50 mF\) can most be measured in circuit, because only a little
measurement voltage of about \(300 mV\) is used.
You should make shure, that all capacitors have no residual charge before starting any measurement.

\end{enumerate}

Thyristors and Triacs can only be detected, if the test current is above the holding current.
Some Thyristors and Triacs need as higher gate trigger current, than this Tester can deliver.
The available testing current is only about 6mA!
Notice that many features can only be used with microcontroller with enough program memory such as ATmega168.
Only processors with at least 32k flash memory like ATmega328 or ATmega1284 can take all features.

\vspace{1cm}
\textbf{{\Large Attention:}} Allways be shure to {\bf discharge capacitors} before connecting them to the Tester!
The Tester may be damaged before you have switched it on. There is only a little protection at the ATmega ports.

Extra causion is required if you try to test components mounted in a circuit.
In either case the equipment should be disconnected from power source and you should be shure,
that {\bf no residual voltage} remains in the equipment.

