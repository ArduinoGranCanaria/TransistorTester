\section*{Preface}
\subsection*{Basically Motive}
Every hobbyist knows the following problem: You disassemle a Transistor out of a printed board 
or you get one out of a collection box. 
If you find out the identification number and you already have a data sheet or you can get the documents about this part,
 everything is well.
But if you don't find any documents, you have no idea, what kind of part this can be.
With conventional approach of measurement it is difficult and time-consuming to find out the type of the part and parameters.
It could be a NPN, PNP, N- or P-Channel-Mosfet etc.
It was the idea of Markus F. to hand over the work to a AVR microcontroller.
\subsection*{As my work has started}
My work with the software of the TransistorTester of Markus F. \cite{Frejek} has started, because I had problems with 
my programmer. I had bought a printed board and components, but I could not program the EEprom of 
the ATmega8 with the Windows driver without error messages. Therefore I took the software of Markus F. and changed all the accesses
from the EEprom memory to flash memory accesses. By analysing the software in order to save memory
at other places of program, I had the idea, to change the result of the ReadADC function from ADC units
to millivolt (mV) units. The mV resolution is needed for any output of voltage values.
If ReadADC returns directly the mV resolution, I can save the conversion for each output value.
This mV resolution can be get, if you first accumulate the results of 22 ADC readings.
 The sum must be multiplied with two and divided by nine. Then we have a maximum value of \begin{math}\frac{1023\cdot22\cdot2}{9} = 5001\end{math},
which matches perfect to the wanted mV resolution of measured voltage values.
So I additionally had the hope, that the enhancement of ADC resolution by oversampling could help to improve the
voltage reading of the ADC, as described in AVR121 \cite{AVR121}.
The original version ReadADC has accumulated the result of 20 ADC measurements and divides afterwards by 20, 
so the result is equal to original ADC resolution. By this way never a enhancement of ADC resolution can take place.
So I had to do little work to change the ReadADC, but this forced analysing the whole program and change
of all ``if-statements'' in the program, where voltage values are queried.
But this was only the beginning of my work!\\

More and more ideas to make measurement faster and more accurate has been implemented.
Additionally the range of resistor and capacity measurements are extended.
The output format for LCD-Display was changed, so symbols are taken for diodes, resistors and capacitors instead of text.
For further details take a look to the actual feature list chapter \ref{sec:features}.
Planned work and new ideas are accumulated in the To Do List in chapter \ref{sec:todo}.
By the way, now I can program the EEprom of the ATmega with Linux operating system without errors.

At this place I would like to thank the originator and software author Markus Frejek, who has enabled the continuation
with his initial work.
In addition I would like to say thanks to the authors of numerous input to the discussion forum, which have assist me, to
find new tasks, weak points and errors.
Next I would like to thank Markus Reschke, who give me the permission, to publish his cheerful software versions at the
SVN server. Furthermore some ideas and software part of Markus R. was integrated in my own software version,
again thank you very much.
Also Wolfgang SCH. has done a great job to support a graphical display with ST7565 controller. Many thanks to him
for integrating his patch for 1.10k software version to the actual developer version.
I have to thank also Asco B., who has developed a new printed board, to enable the reproductions for other hobbyists.
Another thank I would like to send to Dirk W. , who has handled the omnibus order for this printed board.
Never I had time anough to handle these things concurrently with my software developement, at no time the state of further
developement of software would have the same level.
Thanks for the many suggestions to improve the tester to the members of the local chapter of the ''Deutscher Amateur Radio Club (DARC)''
in Lennestadt.
Last but not least I would say thanks to Nick L. from Ukraine, who has supported my with his prototype boards, 
has suggested some hardware add ons and also has organized the russian translation for this description.

