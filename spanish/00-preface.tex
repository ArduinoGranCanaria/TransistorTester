\section*{Prefacio}
\subsection*{Motivación principal}
Todos los aficionados se han enfrentado al siguiente problema: 

Desmontamos un transistor de un circuito o lo tomamos de nuestra colección. Si el número de identificación se puede ver, o se tiene a mano la hoja de especificaciones, no hay ningún problema. Pero si es imposible encontrar una referencia, y no hay documentos, entonces no tenemos ni idea de qué componente se puede tratar. La estrategia habitual de medir para conocer qué tipo de componente es y sus parámetros, con aproximaciones convencionales de mediciones, es complicada y toma tiempo. Podría tratarse de un NPN, PNP, N- o un Mosfet de canal P, etc. 

La idea de dejar toda esa tediosa tarea en manos de un microcontrolador AVR fue de Markus F.

\subsection*{Cómo comencé mi proyecto}

Mi implicación con el TransistorTester de Markus F. [1] comenzó cuando me encontré con problemas con mi programador. Había comprado una placa y unos componentes, pero no podía programar la EEprom del ATmega8 con el controlador de Windows sin obtener mensajes de error. Así que cogí el software de Markus F. y cambié todos los accesos a la memoria EEprom por accesos a la memoria flash. 

Mientras analizaba el software para ahorrar memoria en otras partes del programa, tuve la idea de cambiar el resultado de la función ReadADC de las unidades del conversor analógico-digital a milivoltios (mV). El resultado en mV es necesario en cualquier salida de las medidas del voltaje. Si ReadADC devolviese directamente el resultado en mV, podría guardar la conversión para cada valor de salida. El resultado en mV puede obtenerse acumulando primero los resultados de 22 lecturas del conversor analógico-digital. Luego la suma se multiplica por dos y se divide por nueve. Así tenemos un valor máximo de $\frac{1023·22·2}{9} = 5001$, que coincide perfectamente con el resultado requerido en mV. 

Además, tenía la esperanza de que la mejora en la resolución del conversor analógico-digital por sobremuestreo podría ayudar a mejorar las lecturas del conversor tal y como se describe en AVR121 [5]. La versión original de ReadADC acumulaba el resultado de 20 medidas y las dividía posteriormente por 20, así que el resultado es igual a la resolución original del conversor. Con este método no hay forma de obtener una mejora en la resolución del conversor. Las modificaciones a ReadADC fueron pequeñas, pero para hacerlas me vi obligado a analizar el programa entero y cambiar todas las sentencias "if" del código donde se necesitaban los valores de voltaje. 

¡Y esto sólo fue el comienzo de mi proyecto! Se han implementado muchas más ideas para hacer las medidas más rápidas y más precisas. Además, se ha extendido el rango de las medidas de resistencia y capacidad. Se ha modificado el formato de salida para la pantalla LCD, sustituyendo texto por símbolos para los diodos, resistencias y condensadores. Para más detalles, consulte la lista de características actual en el capítulo 1. El trabajo planificado y las nuevas ideas están acumuladas en el capítulo 9. Por cierto, ahora puedo programar la EEprom del ATmega en un sistema operativo Linux sin errores.

Llegados a este punto me gustaría agradecer al autor original del software Markus Frejek, quien facilitó la continuidad de su trabajo inicial. Además, me gustaría agradecer a los autores de numerosas respuestas en el foros de debate, quienes me han ayudado a encontrar nuevas tareas, puntos débiles y errores. A continuación, me gustaría dar las gracias a Markus Reschke, quien me dio el permiso para publicar las estupendas versiones del software en el servidor SVN. Además, algunas ideas y parte del software de Marks R. se han integrado en mi propia versión, así que, de nuevo, muchas gracias. Wolfgang SCH también ha hecho un trabajo fantástico para soportar una pantalla gráfica con el controlador ST7565. Muchas gracias por integrar su parche de la versión 1.10k en la actual versión de desarrollo. Tengo que agradecer igualmente a Asco B., quien desarrolló un nuevo circuito impreso para que otros aficionados pudieran reproducirla. También me gustaría agradecer a Dirk W. quien gestionó el pedido para imprimir el circuito. Nunca he tenido tiempo suficiente para gestionar estas cosas a la par que el desarrollo de software, y at no time the state of further developement of software would have the same level.

Gracias por las muchas sugerencias para mejorar el multímetro a los miembros del capítulo local del Club de Radioaficionados Alemán (DARC) en Lennestadt. Y por último, querría dar las gracias a Nick L. de Ucrania, que me ha apoyado este prototipo con sus prototipos de circuitos, con sugerencias de componentes y también ha organizado la traducción rusa de esta descripción.

