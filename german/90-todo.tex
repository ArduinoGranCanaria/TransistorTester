
%\newpage
\chapter{Arbeitsliste und neue Ideen}
\label{sec:todo}

\begin{enumerate}
\item Ergänze mehr und bessere Dokumentation.
\item Darüber nachdenken, wie man den wirklichen Innenwiderstand der Port-B-Aus\-gän\-ge (Wider\-stands-Schal\-ter) bestimmen kann,
anstatt anzunehmen, dass die Ports gleich sind.
\item Kann das Entladen von Kondensatoren beschleunigt werden, wenn der Minus-Pol zusätzlich mit dem \(680\Omega\)-
Widerstand nach VCC (+) geschaltet wird?
\item Prüfe, ob der Tester Fließkommadarstellung von Werten gebrauchen kann.
Das Überlaufsrisiko (overflow) ist geringer.
Man braucht keine Konstruktion mit Multiplikation oder Division, um einen Faktor mit einer gebrochenen Zahl nachzubilden.
Aber ich weiß nicht, wieviel Platz für die Bibliothek gebraucht wird.
\item Schreibe eine Gebrauchsanweisung zum Konfigurieren des Testers mit den Makefile-Optionen und beschreibe
den Ablauf bis zum fertigen Prozessor.
\item Wenn der Haltestrom eines Thyristors nicht mit dem \(680\Omega\) Widerstand erreicht werden kann, 
ist es ungefährlich für eine sehr kurze Zeit die Kathode direkt auf GND und die Anode direkt auf VCC zu schalten?
Der Strom kann mehr als 100mA erreichen. Wird der Port beschädigt? Was ist mit der Spannungsversorgung (Spannungsregler)?
\item Prüfe die Ports nach dieser Aktion mit der Selbsttest-Funktion!
\item Idee für ein neues Projekt: USB-Version ohne LC-Display, Power vom USB-Port, Kommunikation zum PC über eine USB-Serial-Brücke.
\item Wählbare getrennte 2-Pin Messung zum schnelleren Selektieren von Bauteilen (Widerstände und Kondensatoren).
\end{enumerate}
